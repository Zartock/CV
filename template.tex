%%%%%%%%%%%%%%%%%%%%%%%%%%%%%%%%%%%%%%%%%
% Twenty Seconds Resume/CV
% LaTeX Template
% Version 1.1 (8/1/17)
%
% This template has been downloaded from:
% http://www.LaTeXTemplates.com
%
% Original author:
% Carmine Spagnuolo (cspagnuolo@unisa.it) with major modifications by 
% Vel (vel@LaTeXTemplates.com)
%
% License:
% The MIT License (see included LICENSE file)
%
%%%%%%%%%%%%%%%%%%%%%%%%%%%%%%%%%%%%%%%%%

%----------------------------------------------------------------------------------------
%	PACKAGES AND OTHER DOCUMENT CONFIGURATIONS
%----------------------------------------------------------------------------------------

\documentclass[letterpaper]{twentysecondcv} % a4paper for A4

%----------------------------------------------------------------------------------------
%	 PERSONAL INFORMATION
%----------------------------------------------------------------------------------------

% If you don't need one or more of the below, just remove the content leaving the command, e.g. \cvnumberphone{}

\profilepic{Konsultbild.jpg} % Profile picture

\cvname{Viktor Karlsson} % Your name
\cvjobtitle{Software developer} % Job title/career

\cvdate{2 August 1995} % Date of birth
\cvaddress{Grankottevägen 6,\newline 70282, Sweden} % Short address/location, use \newline if more than 1 line is required
\cvnumberphone{+46 762665056} % Phone number
\cvsite{https://github.com/Zartock} % Personal website
\cvmail{virrevickan95@hotmail.com} % Email address

%----------------------------------------------------------------------------------------

\begin{document}

%----------------------------------------------------------------------------------------
%	 ABOUT ME
%----------------------------------------------------------------------------------------

\aboutme
{
	A young and hungry developer with the goal to learn about and work with 			technologies that are at the forefront of today's development.
} % To have no About Me section, just remove all the text and leave \aboutme{}

%----------------------------------------------------------------------------------------
%	 SKILLS
%----------------------------------------------------------------------------------------

% Skill bar section, each skill must have a value between 0 an 6 (float)
\skills{{C++/3.5},{C\#/3},{ASP.NET/2.5},{Xamarin/2},{Python/2},{Java/0.01}, {SQL/0.02}}

%----------------------------------------------------------------------------------------
%	 Hobbies
%----------------------------------------------------------------------------------------

\hobbies
{
	Gaming and music
}

%------------------------------------------------

% Skill text section, each skill must have a value between 0 an 6
\skillstext{}

%----------------------------------------------------------------------------------------

\makeprofile % Print the sidebar

%----------------------------------------------------------------------------------------
%	 INTERESTS
%----------------------------------------------------------------------------------------

\section{Interests}

Full stack development, XR, App development, Backend development and Cloud

%----------------------------------------------------------------------------------------
%	 EDUCATION
%----------------------------------------------------------------------------------------

\section{Education}

\begin{twenty} % Environment for a list with descriptions
	\twentyitem{2014-2017}{B.Sc. {\normalfont in Computer Science}}{Örebro University}{\emph{A Quantified Theory of Social Cohesion.}}
	\twentyitem{2011-2014}{High school}{Grillska Gymnasiet Örebro}{Specializing in information- and media technology}
	%\twentyitem{<dates>}{<title>}{<location>}{<description>}
\end{twenty}

%----------------------------------------------------------------------------------------
%	 PUBLICATIONS
%----------------------------------------------------------------------------------------

\section{Languages}

\begin{twentyshort} % Environment for a short list with no descriptions
	\twentyitemshort{MT*}{Swedish}
	\twentyitemshort{Fluent}{English}
	\twentyitemshort{Beginner}{Spanish}
	\twentyitemshort{Beginner}{Japanese}
	%\twentyitemshort{<dates>}{<title/description>}
\end{twentyshort}

\footnotesize (*)[MT = Mother tounge]

%----------------------------------------------------------------------------------------
%	 EXPERIENCE
%----------------------------------------------------------------------------------------

\section{Experience}

\begin{twenty} % Environment for a list with descriptions
	\twentyitem{2017-}{Consultant @ Sogeti Örebro}{IT}
	{
		Organizing and executing assigned business projects on behalf of clients 			according to client's requirements. \emph{(See other experiences during 			the same time period)}\newline
		Since fall 2018: Supervisor for students doing their thesis projects at 			Sogeti Örebro. \newline
		Since late 2018: Responsible for Sogeti Örebro's competence development 			with focus towards software development.
	}
	\twentyitem{2017-}{Software developer @ Epiroc}{IT}
	{
		Software developer at Epiroc with focus on automation.\newline
		Automation and maintenance of Epiroc's Pit Viper machines' software, 				as well as test automation.
	}
	\twentyitem{2018-}{AIO @ Alreadit AB}{IT}
	{
		Assistant Information Officer at Alreadit AB. Work includes developing 				and maintaining their mobile application as well the overall IT-					infrastructure.
	}
	%\twentyitem{<dates>}{<title>}{<location>}{<description>}
\end{twenty}

%----------------------------------------------------------------------------------------
%	 OTHER INFORMATION
%----------------------------------------------------------------------------------------

\section{Other information}

\subsection{Review}

Alice approaches Wonderland as an anthropologist, but maintains a strong sense of noblesse oblige that comes with her class status. She has confidence in her social position, education, and the Victorian virtue of good manners. Alice has a feeling of entitlement, particularly when comparing herself to Mabel, whom she declares has a ``poky little house," and no toys. Additionally, she flaunts her limited information base with anyone who will listen and becomes increasingly obsessed with the importance of good manners as she deals with the rude creatures of Wonderland. Alice maintains a superior attitude and behaves with solicitous indulgence toward those she believes are less privileged.

%----------------------------------------------------------------------------------------
%	 SECOND PAGE EXAMPLE
%----------------------------------------------------------------------------------------

%\newpage % Start a new page

%\makeprofile % Print the sidebar

%\section{Other information}

%\subsection{Review}

%Alice approaches Wonderland as an anthropologist, but maintains a strong sense of noblesse oblige that comes with her class status. She has confidence in her social position, education, and the Victorian virtue of good manners. Alice has a feeling of entitlement, particularly when comparing herself to Mabel, whom she declares has a ``poky little house," and no toys. Additionally, she flaunts her limited information base with anyone who will listen and becomes increasingly obsessed with the importance of good manners as she deals with the rude creatures of Wonderland. Alice maintains a superior attitude and behaves with solicitous indulgence toward those she believes are less privileged.

%\section{Other information}

%\subsection{Review}

%Alice approaches Wonderland as an anthropologist, but maintains a strong sense of noblesse oblige that comes with her class status. She has confidence in her social position, education, and the Victorian virtue of good manners. Alice has a feeling of entitlement, particularly when comparing herself to Mabel, whom she declares has a ``poky little house," and no toys. Additionally, she flaunts her limited information base with anyone who will listen and becomes increasingly obsessed with the importance of good manners as she deals with the rude creatures of Wonderland. Alice maintains a superior attitude and behaves with solicitous indulgence toward those she believes are less privileged.

%----------------------------------------------------------------------------------------

\end{document} 
